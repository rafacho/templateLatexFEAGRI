\documentclass[12pt,oneside,hidelinks,a4paper,final]{book}
	% Font size (10pt, 11pt, 12pt)
	% Paper size and format (a4paper, letterpaper, etc.)
	% Draft mode (draft) -
	% Multiple columns (onecolumn, twocolumn)
	% Formula-specific options
		%fleqn -  left alignment of formulas
		%leqno -  labels formulas on the left hand side instead of right
	% Landscape print mode (landscape)
	% Single and double sided documents (oneside, twoside)
	% Titlepage behavior (notitlepage, titlepage)
	% Chapter opening page (openright, openany) - Faz com que os capítulos comecem somente nas páginas da direita ou na próxima página disponível.


% FONTES
	% \usepackage{fontspec} % caracteres especiais para lualatex
	\usepackage[T1]{fontenc}
	\usepackage[utf8]{inputenc} %Permite a especificação das tabelas de caracteres ASCII, ISO Latin-1, ISO Latin-2, páginas de código 437/850 IBM, Apple Macintosh, Next, ANSI-Windows ou alguma tabela definida pelo usuário.
	\usepackage{ae} %Seta as fontes
	\usepackage{times} %Define tipo da fonte como "Times new roman"

% LÍNGUA
	\usepackage[portuguese,brazil]{babel} %Pacote de tradução	
	% \usepackage[american]{babel} %Pacote de tradução

% MARGENS 
	\usepackage[top=30mm,left=30mm,right=20mm,bottom=20mm,marginparsep=0pt,marginparwidth=0pt,headsep=0pt,headheight=0pt,voffset = 0pt,hoffset = 0pt,marginparsep = 0pt]{geometry}
	%\usepackage[paperwidth=8.5in,paperheight=11in,hmargin={30mm,25mm},vmargin={25mm,25mm}]{geometry} 
	\setlength{\headheight}{15pt} % espaço para heading no topo da página 

% Suprime a numeração da lista de tabelas e figuras
	\usepackage{tocloft}

% CITAÇÕES DIRETAS
	\renewenvironment{quote}
		{\small\list{}{\rightmargin=0cm \leftmargin=4cm}%
		\item\relax}
		{\endlist}


% ORIENTAÇÃO
	\usepackage{pdflscape} % Muda a orientação da página retrato / paisagem

% PARÁGRAFOS/LINHAS
	\usepackage[onehalfspacing]{setspace} % Libera a edição do espaçamento entre linhas
						  % No corpo do texto usar:
	    						% \singlespacing - Para um espaçamento simples
	    						% \onehalfspacing - Para um espaçamento de 1,5
	    						% \doublespacing - Para um espaçamento duplo
	\usepackage{indentfirst} % Identa a primeira palavra do primeiro parágrafo da seção
	\setlength{\parskip}{0cm} % espaço entre parágrafos
	\setlength\parindent{2cm}	% primeira linha do parágrafo tem recuo de 2cm

% Inclui numeração de seções em \susubsection e \paragraph
	\setcounter{secnumdepth}{5} % Note that part is -1 level !
	\setcounter{tocdepth}{5}
	

% CONTROLE DE LINHAS ORFÃS E VIUVAS
	\clubpenalty=10000
	\widowpenalty=10000

% REFERÊNCIAS
	% \usepackage{currfile-abspath}
	% \usepackage[alf]{abntex2cite}
	% \usepackage[comma]{natbib}
		
    \usepackage[style=abnt, justify]{biblatex}
    	\addbibresource{bibliografia.bib}
        \bibliography{bibliografia.bib}

% CITAÇÃO
	% \newcommand\posscite[1]{\citeauthor{#1}'s \citeyear{#1}} %para possessive citation em ingles
	% \let\citep\cite
 								
% OBJETOS
	\usepackage{float}
	\usepackage{amssymb, amsmath} % Pacotes matemáticos
	\usepackage{fancyhdr,fancybox,epsfig,psfrag,tabularx}
	

% FIGURES
	\usepackage{pgf}			% Habilita figuras em pgf
	\usepackage{pgfplots}
	\usepackage{graphicx}		% Habilita figuras
	\graphicspath{{Figures/}}	% Define pasta onde estão as figuras

	\usepgfplotslibrary{external}  % Compila as figuras separadamente


	% pacote mais atual que o subfigure
	\usepackage[%
    font={small,sf},
    labelfont=bf,
    format=hang,    
    format=plain,
    margin=0pt,
    width=0.8\textwidth,    
	]{caption}
	\usepackage[list=true]{subcaption}
	% \usepackage[position=top]{subfig}



% TABLES
	\usepackage{longtable}          % pacote que quebra a tabela em mais de uma página
	\usepackage{booktabs}        % for well-spaced horizontal rules
	\usepackage{threeparttablex} 	% para colocar notas em longtable
	\usepackage{multicol} % multicolunas
	\usepackage{multirow} % multilinhas
	\usepackage{hhline}

% LINKS
	\usepackage{hyperref}			% para ter hyperlink dos capítulos (deletar arquivos .AUX e rodar duas vezes)
	% \usepackage[all]{hypcap}        % soluciona o problema com o hyperref e capitulos
		\hypersetup{colorlinks=true,bookmarksnumbered,anchorcolor=black,citecolor=black,linkcolor=black,bookmarksopen=false,pdfstartview={FitV}} % links	em preto de referencia cruzadas e da bibliografia; também faz que o pdf seja aberto com a aba das seções e que ocupe a pagina inteira

	\usepackage{bookmark} 	% This package implements a new bookmark (outline) organization for package hyperref.


%FOOTNOTES
	\usepackage[hang,flushmargin]{footmisc} 	% para não haver espaçamento antes das notas de rodapé


% VISUAL DO PDF
	\usepackage{microtype}  % melhora a visualização do texto (faz que os espaçamentos do texto seja mais homogeneo).

% NUMERAÇÃO DE PÁGINAS
	% para colocar número de páginas em cima à direita
	\pagestyle{fancy}
		\fancyhf{}
		\fancyheadoffset{0cm}
		\renewcommand{\headrulewidth}{0pt} 
		\renewcommand{\footrulewidth}{0pt}
		\fancyhead[R]{\thepage}
		\fancypagestyle{plain}{%
		  \fancyhf{}%		
		  \fancyhead[R]{\thepage}%
		}

% OPÇÕES PARA FORMATO BOOK
	%\usepackage[Lenny]{fncychap} % coloca um quadrado ao redor do "capítulo 1"	
	% \setlength{\headheight}{15pt}

% CORES
	\usepackage[color]{showkeys}
	\definecolor{refkey}{rgb}{0.39,0.58,1}
	\definecolor{labeled}{rgb}{1,0,0}

% Flow images
	\usepackage{tikz}
	% \tikzexternalize	
	\usetikzlibrary{shapes,arrows}
	% \usetikzlibrary{snakes}
	\usetikzlibrary{calc}
	\tikzset{
	  -|-/.style={
	    to path={
	      (\tikztostart) -| ($(\tikztostart)!#1!(\tikztotarget)$) |- (\tikztotarget)
	      \tikztonodes}
	  },
	  -|-/.default=0.5,
	  |-|/.style={
	    to path={
	      (\tikztostart) |- ($(\tikztostart)!#1!(\tikztotarget)$) -| (\tikztotarget)
	      \tikztonodes}
	  },
	  |-|/.default=0.5,
	}
	\tikzset{
	    charge node/.style={inner sep=0pt},
	    pics/sum block/.style n args={4}{
	      code={
	        \path node (n) [draw, circle, inner sep=0pt, minimum size=9mm] {}
	          (n.north) +(0,-1.5mm) node [charge node] {$#1$}
	          (n.south) +(0,1.5mm) node [charge node] {$#2$}
	          (n.west) +(1.5mm,0) node [charge node] {$#3$}
	          (n.east) +(-1.5mm,0) node [charge node] {$#4$}
	          ;}}}\usetikzlibrary{tikzmark,calc}
\usepackage{ifthen}
\usepackage{helvet,verbatim}

% plots in latex
	\usepackage{pgfplots}
  		\pgfplotsset{compat=1.3}


% to include pdf pages
	\usepackage{pdfpages}

% to prevent figures for appearing before the section wherein they are declared
	\usepackage[section]{placeins}

% This package provides user control over the layout of the three basic list environments: enumerate, itemize and description
	\usepackage{enumitem}

% This package allows to format code
% For config information access https://github.com/gpoore/minted/blob/master/source/minted.pdf
	% \usepackage{minted}
	% Set the options for the document as a whole
	% \definecolor{bg}{rgb}{0.95, 0.95, 0.95} % Code background color
	% \setminted[xml]{linenos, 
	% 				bgcolor=bg,
	% 				autogobble, 
	% 				% gobble=5, 
	% 				samepage, 
	% 				tabsize=2,					 
	% 				breaklines,
	% 				breakautoindent,
	% 				breakanywhere,
	% 				bgcolor=bg,
	% 				}

	\usepackage{csquotes}

	% \usepackage[some]{background} % insere marca d'água na página
	% \SetBgScale{6}
	% \SetBgColor{gray}
	% \SetBgOpacity{0.1}
	% \backgroundsetup{contents=\includegraphics{Apresentacao/image/capa/feagri.png}}
	% \SetBgAngle{0}

	\usepackage{epigraph}	% Formata epígrafe
	